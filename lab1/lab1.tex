% !TEX encoding = UTF-8 Unicode
\documentclass[10pt,a4paper]{article}
\usepackage[T1]{fontenc}
\usepackage[brazil]{babel}
%\usepackage[utf8]{inputenc}


\usepackage{ae,aecompl}
\usepackage{pslatex}
\usepackage{epsfig}
\usepackage{geometry}
\usepackage{url}
\usepackage{textcomp}
\usepackage{ae}
\usepackage{subfig}
\usepackage{indentfirst}
\usepackage{textcomp}
\usepackage{color}
\usepackage{setspace}
\usepackage{verbatim}


% Definindo as margens para 2cm e o espaçamento entre linhas para 1.5
% Relatório parcial deve ter espaçamento simples
% \linespread{1.5}

\geometry{ 
  a4paper,	% Formato do papel
  tmargin=40mm,	% Margem superior
  bmargin=20mm,	% Margem inferior
  lmargin=20mm,	% Margem esquerda
  rmargin=20mm,	% Margem direita
  footskip=20mm	% Espaço entre o rodapé e o fim do texto
}
\include{abaco} 
\renewcommand{\thetable}{\Roman{table}}
\newcommand{\x} {$\bullet$}


\begin{document}

% CAPA
\begin{titlepage}
  \thispagestyle{empty}
  \begin{center} {\large \textbf{UNIVERSIDADE~ESTADUAL~DE~CAMPINAS}} \end{center}
  \begin{center} {\large INSTITUTO~DE~COMPUTAÇÃO}                    \end{center}
  \vspace{0.1cm}
  \begin{center}
    \begin{minipage}[tl]{31mm}
      \ABACO{1}{9}{6}{9}{1}
    \end{minipage}
  \end{center}
  \vspace{0.3cm}
  \begin{center} 
    {\large \textsc{Servidor de Agenda baseado em socket TCP
      }} 
    \\\vspace{0.5cm}
    {\textsl{Relatório do primeiro laboratório de MC823}}
    \\\vspace{1cm}
    \begin{tabular}{ll}
      \textbf{Aluno}:        Marcelo~Keith~Matsumoto   &  \textbf{RA}:       085937 \\
   %   \textbf{Turma}:       & A \\
      \textbf{Aluno}:        Tiago~Chedraoui~Silva    &   \textbf{RA}:       082941 \\
   
    \end{tabular}
  \end{center}
  \vspace{0.5cm}

  \begin{abstract}

  \end{abstract}

  % Sum√°rio
  \tableofcontents
\end{titlepage} 



% -----------------------------------------------------------------------------%
\section{Introdução}
% -----------------------------------------------------------------------------%
	Este laboratório tem o objetivo de medir o tempo e eficiência de uma comunicação entre cliente e servidor num protocolo TCP, que requer uma conexão.

% -----------------------------------------------------------------------------%
\section{Servidor de agenda}
% -----------------------------------------------------------------------------%
	O sistema implementado se baseia numa comunicação cliente-servidor. O cliente possui todas as informações da agenda, assim como a estrutura dos menus. O cliente só escolhe alguma opção do menu e insere as informações de um compromisso, como nome, dia, hora e minuto.

\section{Ambiente de implementação}

\section{Tempos de comunicação}   %Verificar nome desta seção!

\section{Conclus√£o}

% ******************************************************
% REFERENCIAS BIBLIOGRÁFICAS
% ******************************************************
% \section{Referências}
\bibliographystyle{plain}
\begin{small}
  \bibliography{referencias}
\end{small}

\end{document}
