\documentclass[10pt,a4paper]{article}
\usepackage[T1]{fontenc}
\usepackage[brazil]{babel}
\usepackage[utf8]{inputenc}


\usepackage{ae,aecompl}
\usepackage{pslatex}
\usepackage{epsfig}
\usepackage{geometry}
\usepackage{url}
\usepackage{textcomp}
\usepackage{ae}
\usepackage{subfig}
\usepackage{indentfirst}
\usepackage{textcomp}
\usepackage{color}
\usepackage{setspace}
\usepackage{verbatim}


% Definindo as margens para 2cm e o espaçamento entre linhas para 1.5
% Relatório parcial deve ter espaçamento simples
 \linespread{1.5}

\geometry{ 
  a4paper,	% Formato do papel
  tmargin=40mm,	% Margem superior
  bmargin=20mm,	% Margem inferior
  lmargin=20mm,	% Margem esquerda
  rmargin=20mm,	% Margem direita
  footskip=20mm	% Espaço entre o rodapé e o fim do texto
}
%  ABACO -- Conjunto de macros para desenhar o 'abaco

%  Desenho original de Hans Liesenberg

%  Macros de Tomasz Kowaltowski

%  DCC -- IMECC -- UNICAMP

%  Mar,co de 1988  --  Vers~ao 1.0

% Ajustado para LaTeX da SUN -- Mar,co de 1991

% ---------------------------------------------------------

%  Chamada:   \ABACO{d1}{d2}{d3}{d4}{esc}
%             com:  di's -- os quatro d'igitos;
%	           esc  -- fator de escala

% ---------------------------------------------------------

%  DEFINI,C~OES AUXILIARES

% ---------------------------------------------------------


%  Forma o d'igito pequeno (0 ou 1)

\newcommand{\ABACODP}[1]{%
%
\thicklines
%    
\begin{picture}(8,0)
    \ifcase#1{   %  caso 0
       \put(0,0)    {\line(1,0){4}}
       \multiput(5,0)(2,0){2}{\oval(2,4)}}
    \or{         %  caso 1
       \put(2,0)    {\line(1,0){4}}
       \multiput(1,0)(6,0){2}{\oval(2,4)}}
    \fi
\end{picture}
    } % \ABACODP

% Forma o d'igito grande (0 a 4)

\newcommand{\ABACODG}[1]{%
%
\thicklines
%    
\begin{picture}(14,0)
    \ifcase#1{   % caso 0
       \multiput(1,0)(2,0){5}{\oval(2,4)}}
       \put(10,0)   {\line(1,0){4}}
    \or{         % caso 1
       \multiput(1,0)(2,0){4}{\oval(2,4)}}
       \put(8,0)   {\line(1,0){4}}
       \put(13,0)   {\oval(2,4)}
    \or{         % caso 2
       \multiput(1,0)(2,0){3}{\oval(2,4)}
       \put(6,0)   {\line(1,0){4}}
       \multiput(11,0)(2,0){2}{\oval(2,4)}}
    \or{         % caso 3
       \multiput(1,0)(2,0){2}{\oval(2,4)}
       \put(4,0)   {\line(1,0){4}}
       \multiput(9,0)(2,0){3}{\oval(2,4)}}
    \or{         % caso 4
       \put(1,0)  {\oval(2,4)}}
       \put(2,0)   {\line(1,0){4}}
       \multiput(7,0)(2,0){4}{\oval(2,4)}
    \fi
\end{picture}
    } % \ABACODG
       
% Forma um d'igito (0 a 9)

\newcommand{\ABACOD}[1]{%
%
    \ifnum#1>9
       \errmessage{#1: Argumento invalido para ABACO}
    \fi
    \ifnum#1<0
       \errmessage{#1: Argumento invalido para ABACO}
    \fi
%
\begin{picture}(24,0)
%    
    \ifnum#1<5
       \put(16,0) {\ABACODP{0}}
    \else   
       \put(16,0) {\ABACODP{1}}
    \fi
%    
    \ifnum#1<5
       \put(0,0)  {\ABACODG{#1}}
    \else
       \ifcase#1\or \or \or \or
          \or  \put(0,0)  {\ABACODG{0}}
          \or  \put(0,0)  {\ABACODG{1}}
          \or  \put(0,0)  {\ABACODG{2}}
          \or  \put(0,0)  {\ABACODG{3}}
          \or  \put(0,0)  {\ABACODG{4}}
       \fi
    \fi   
\end{picture}
    } % \ABACOD
    
% -------------------------------------------------

%  DEFINI,C~AO PRINCIPAL
    
\newcommand{\ABACO}[5]{%
    \setlength{\unitlength}{#5mm}
%
    \thinlines
%   
\begin{picture}(28,25)
%   
% moldura
%
% externa
%
        \put(0,0)            {\line(0,1){25}}
        \put(0,0)            {\line(1,0){28}}
        \put(28,0)           {\line(0,1){25}}
        \put(0,25)           {\line(1,0){28}}
% interna
        \put(2,2)            {\line(0,1){21}}
	\put(26,2)           {\line(0,1){21}}
	\put(16,2)           {\line(0,1){21}}
	\put(18,2)           {\line(0,1){21}}
	\put(2,2)            {\line(1,0){14}}
	\put(16,2)           {\line(1,-1){1}}
	\put(17,1)           {\line(1,1){1}}
	\put(18,2)           {\line(1,0){8}}
	\put(2,23)           {\line(1,0){14}}
	\put(16,23)          {\line(1,1){1}}
	\put(17,24)          {\line(1,-1){1}}
	\put(18,23)          {\line(1,0){8}}
	\put(0,0)            {\line(1,1){2}}
	\put(0,25)           {\line(1,-1){2}}
	\put(28,0)           {\line(-1,1){2}}
	\put(28,25)          {\line(-1,-1){2}}
%
%   
% d'igitos
%
%   
       \put(2,20)  {\ABACOD{#1}}
       \put(2,15)  {\ABACOD{#2}}
       \put(2,10)  {\ABACOD{#3}}
       \put(2,5)   {\ABACOD{#4}}
%      
\end{picture}
    } % \ABACO
    
 
\renewcommand{\thetable}{\Roman{table}}
\newcommand{\x} {$\bullet$}


\begin{document}

% CAPA
\begin{titlepage}
  \thispagestyle{empty}
  \begin{center} {\large \textbf{UNIVERSIDADE~ESTADUAL~DE~CAMPINAS}} \end{center}
  \begin{center} {\large INSTITUTO~DE~COMPUTAÇÃO}                    \end{center}
  \vspace{0.1cm}
  \begin{center}
    \begin{minipage}[tl]{31mm}
      \ABACO{1}{9}{6}{9}{1}
    \end{minipage}
  \end{center}
  \vspace{0.3cm}
  \begin{center} 
    {\large \textsc{Servidor de Agenda baseado em socket TCP
      }} 
    \\\vspace{0.5cm}
    {\textsl{Relatório do primeiro laboratório de MC823}}
    \\\vspace{1cm}
    \begin{tabular}{ll}
      \textbf{Aluno}:        Marcelo~Keith~Matsumoto   &  \textbf{RA}:       085937 \\
   %   \textbf{Turma}:       & A \\
      \textbf{Aluno}:        Tiago~Chedraoui~Silva    &   \textbf{RA}:       082941 \\
   
    \end{tabular}
  \end{center}
  \vspace{0.5cm}

  \begin{abstract}

  \end{abstract}

  % Sumário
  \tableofcontents
\end{titlepage} 



% -----------------------------------------------------------------------------%
\section{Introdução}
% -----------------------------------------------------------------------------%
O TCP (Transmission Control Protocol) é um protocolo do nível da
camada de transporte. Dentre usas principais características temos:
\begin{description}
\item[Orientado à conexão]  Cliente e o servidor trocam pacotes de
controle entre si antes de enviarem os pacotes de dados. Isto é chamado de procedimento de
estabelecimento de conexão (handshaking). 
\item[Transferência garantida] Dados trocados são livres de erro, o que é conseguido a partir de
mensagens de reconhecimento e retransmissão de pacotes. 
\item[Controle de fluxo] Assegura que nenhum dos lados da comunicação envie pacotes rápido demais, pois uma aplicação em um
lado pode não conseguir processar a informação na velocidade que está
recebendo.
\item[controle de congestão] Ajuda a prevenir congestionamentos na rede.


\end{description}


  Este laboratório tem o objetivo de medir o tempo total e de comunicação de uma conexão TCP entre um cliente e um servidor.

% -----------------------------------------------------------------------------%
\section{Servidor de agenda}
% -----------------------------------------------------------------------------%
  O sistema implementado se baseia numa comunicação cliente-servido O cliente possui todas as informações da agenda, assim como a estrutura dos menus. O cliente só escolhe alguma opção do menu e insere as informações de um compromisso, como nome, dia, hora e minuto.

% -----------------------------------------------------------------------------%
\section{Ambiente de implementação}
% -----------------------------------------------------------------------------%
  O sistema de agenda foi implementado na linguagem C. Os dados da agenda foram armazenados em arquivos, onde o servidor lê quando um usuário loga no sistema de agenda e os armazena em memória. A cada alteração na agenda o servidor atualiza as informações dos arquivos.

\section{Tempos de comunicação e total}
O round-trip time (RTT) é o tempo que leva-se para um sinal ser
enviado mais o tempo que se leva para receber um acknowledgment que o
sinal foi recebido. A ferramenta administrativa para as redes de
computadores denominada ``Ping'' é usada para testar se um host é alcançável e para
medir o RTT para mensagens enviadas do host remetente para o
destinatário.

Inicialmente, implementamos um programa semelhante ao ping para o
cálculo da RTT. Com ele foi possível calcular várias vezes o tempo de envio pacote
de 1 byte para o servidor e esse responder com um pacote de 4
bytes. Utilizando um script para a coleta dos tempo, obtivemos o
seguintes valores:

\begin{table}[h!]
\caption{Ping implementado}
\begin{center}
  \begin{tabular}{lr}
    \multicolumn{1}{c}{Valor} & \multicolumn{1}{c}{Tempo}\\
    \hline
    Max & 17.814 ms\\
    Min & 0.045 ms\\
    Média & 0.059 ms \\
    Desvio & 0.232 ms
  \end{tabular}

\end{center}
\end{table}

Posteriormente, aplicamos o cálculo de tempo ao programa principal de
forma a obtermos o tempo total e tempo de comunicação.
Para o tempo total, no cliente pega-se  o tempo antes do primeiro send e após o último recv.
Para o tempo de comunicação, no servidor pega-se o tempo após o
primeiro recv e antes do último send.
Subtraindo o segundo do primeiro você tem o tempo estimado de ida e
volta de um pacote. 

O resultado obtido para 100 valores foi:

\begin{table}[h!]
\caption{Teste 1: conexão e fechamento de conexão com servidor}
\begin{center}
  \begin{tabular}{lr}
    \multicolumn{1}{c}{Valor} & \multicolumn{1}{c}{Tempo}\\
    \hline
    Max & 4.986 ms\\
    Min & 0.097 ms\\
    Média & 0.608 ms \\
    Desvio & 0.008 ms
  \end{tabular}

\end{center}
\end{table}

\begin{table}[h!]
\caption{Teste 2: conexão, login na conta, ver agenda do mês e fechamento de conexão com servidor}
\begin{center}
  \begin{tabular}{lr}
    \multicolumn{1}{c}{Valor} & \multicolumn{1}{c}{Tempo}\\
    \hline
    Max & 4.959 ms\\
    Min & 0.190 ms\\
    Média & 0.638 ms \\
    Desvio & 0.013 ms
  \end{tabular}

\end{center}
\end{table}

Comparando os resultados, como o ping é muito rápido nossa precisão é
muito baixa. Porém nosso programa não é tão rápido assim os valores
são mais precisos. Para certificarmos do tmepo, fizemos o teste com
três situações, uma mais rápida, uma mais lenta. E obtemos valores
muito próximos da média de tempo de ida e volta de um pacote(RTT). 

\section{Conclusão}

% ******************************************************
% REFERENCIAS BIBLIOGRÁFICAS
% ******************************************************
% \section{Referências}
\bibliographystyle{plain}
\begin{small}
  \bibliography{referencias}
\end{small}

\end{document}
