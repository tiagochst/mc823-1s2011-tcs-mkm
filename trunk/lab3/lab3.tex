\documentclass[10pt,a4paper]{article}
\usepackage[T1]{fontenc}
\usepackage[brazil]{babel}
\usepackage[utf8]{inputenc}

\newcommand{\red}[1]{\textcolor{red}{#1}}

\usepackage{ae,aecompl}
\usepackage{pslatex}
\usepackage{epsfig}
\usepackage{geometry}
\usepackage{url}
\usepackage{textcomp}
\usepackage{ae}
\usepackage{subfig}
\usepackage{indentfirst}
\usepackage{textcomp}
\usepackage{color}
\usepackage{setspace}
\usepackage{verbatim}
\usepackage{multicol}

\usepackage{listings}
\lstset{language=C}
\usepackage{listings}
\lstset{
  basicstyle=\footnotesize\ttfamily, % Standardschrift
  numbers=left,               % Ort der Zeilennummern
  numberstyle=\tiny,          % Stil der Zeilennummern
  % stepnumber=2,               % Abstand zwischen den Zeilennummern
  numbersep=5pt,              % Abstand der Nummern zum Text
  tabsize=2,                  % Groesse von Tabs
  extendedchars=true,         %
  breaklines=true,            % Zeilen werden Umgebrochen
  classoffset=0,
  keywordstyle=\color{blue},
  classoffset=1,
  morekeywords={factor},keywordstyle=\color{red},
  classoffset=0,
  % frame=b,         
  % keywordstyle=[1]\textbf,    % Stil der Keywords
  % keywordstyle=[2]\textbf,    %
  % keywordstyle=[3]\textbf,    %
  % keywordstyle=[4]\textbf,   \sqrt{\sqrt{}} %
  stringstyle=\color{green}\ttfamily, % Farbe der String
  showspaces=false,           % Leerzeichen anzeigen ?
  showtabs=false,             % Tabs anzeigen ?
  xleftmargin=17pt,
  framexleftmargin=17pt,
  framexrightmargin=5pt,
  framexbottommargin=4pt,
  % backgroundcolor=\color{lightgray},
  showstringspaces=false      % Leerzeichen in Strings anzeigen ?        
}
\lstloadlanguages{C
}

% Definindo o espaçamento entre linhas para 1.5
% Relatório parcial deve ter espaçamento simples
\linespread{1.5}

\geometry{ 
  a4paper,	% Formato do papel
  tmargin=30mm,	% Margem superior
  bmargin=30mm,	% Margem inferior
  lmargin=20mm,	% Margem esquerda
  rmargin=20mm,	% Margem direita
  footskip=20mm	% Espaço entre o rodapé e o fim do texto
}
%  ABACO -- Conjunto de macros para desenhar o 'abaco

%  Desenho original de Hans Liesenberg

%  Macros de Tomasz Kowaltowski

%  DCC -- IMECC -- UNICAMP

%  Mar,co de 1988  --  Vers~ao 1.0

% Ajustado para LaTeX da SUN -- Mar,co de 1991

% ---------------------------------------------------------

%  Chamada:   \ABACO{d1}{d2}{d3}{d4}{esc}
%             com:  di's -- os quatro d'igitos;
%	           esc  -- fator de escala

% ---------------------------------------------------------

%  DEFINI,C~OES AUXILIARES

% ---------------------------------------------------------


%  Forma o d'igito pequeno (0 ou 1)

\newcommand{\ABACODP}[1]{%
%
\thicklines
%    
\begin{picture}(8,0)
    \ifcase#1{   %  caso 0
       \put(0,0)    {\line(1,0){4}}
       \multiput(5,0)(2,0){2}{\oval(2,4)}}
    \or{         %  caso 1
       \put(2,0)    {\line(1,0){4}}
       \multiput(1,0)(6,0){2}{\oval(2,4)}}
    \fi
\end{picture}
    } % \ABACODP

% Forma o d'igito grande (0 a 4)

\newcommand{\ABACODG}[1]{%
%
\thicklines
%    
\begin{picture}(14,0)
    \ifcase#1{   % caso 0
       \multiput(1,0)(2,0){5}{\oval(2,4)}}
       \put(10,0)   {\line(1,0){4}}
    \or{         % caso 1
       \multiput(1,0)(2,0){4}{\oval(2,4)}}
       \put(8,0)   {\line(1,0){4}}
       \put(13,0)   {\oval(2,4)}
    \or{         % caso 2
       \multiput(1,0)(2,0){3}{\oval(2,4)}
       \put(6,0)   {\line(1,0){4}}
       \multiput(11,0)(2,0){2}{\oval(2,4)}}
    \or{         % caso 3
       \multiput(1,0)(2,0){2}{\oval(2,4)}
       \put(4,0)   {\line(1,0){4}}
       \multiput(9,0)(2,0){3}{\oval(2,4)}}
    \or{         % caso 4
       \put(1,0)  {\oval(2,4)}}
       \put(2,0)   {\line(1,0){4}}
       \multiput(7,0)(2,0){4}{\oval(2,4)}
    \fi
\end{picture}
    } % \ABACODG
       
% Forma um d'igito (0 a 9)

\newcommand{\ABACOD}[1]{%
%
    \ifnum#1>9
       \errmessage{#1: Argumento invalido para ABACO}
    \fi
    \ifnum#1<0
       \errmessage{#1: Argumento invalido para ABACO}
    \fi
%
\begin{picture}(24,0)
%    
    \ifnum#1<5
       \put(16,0) {\ABACODP{0}}
    \else   
       \put(16,0) {\ABACODP{1}}
    \fi
%    
    \ifnum#1<5
       \put(0,0)  {\ABACODG{#1}}
    \else
       \ifcase#1\or \or \or \or
          \or  \put(0,0)  {\ABACODG{0}}
          \or  \put(0,0)  {\ABACODG{1}}
          \or  \put(0,0)  {\ABACODG{2}}
          \or  \put(0,0)  {\ABACODG{3}}
          \or  \put(0,0)  {\ABACODG{4}}
       \fi
    \fi   
\end{picture}
    } % \ABACOD
    
% -------------------------------------------------

%  DEFINI,C~AO PRINCIPAL
    
\newcommand{\ABACO}[5]{%
    \setlength{\unitlength}{#5mm}
%
    \thinlines
%   
\begin{picture}(28,25)
%   
% moldura
%
% externa
%
        \put(0,0)            {\line(0,1){25}}
        \put(0,0)            {\line(1,0){28}}
        \put(28,0)           {\line(0,1){25}}
        \put(0,25)           {\line(1,0){28}}
% interna
        \put(2,2)            {\line(0,1){21}}
	\put(26,2)           {\line(0,1){21}}
	\put(16,2)           {\line(0,1){21}}
	\put(18,2)           {\line(0,1){21}}
	\put(2,2)            {\line(1,0){14}}
	\put(16,2)           {\line(1,-1){1}}
	\put(17,1)           {\line(1,1){1}}
	\put(18,2)           {\line(1,0){8}}
	\put(2,23)           {\line(1,0){14}}
	\put(16,23)          {\line(1,1){1}}
	\put(17,24)          {\line(1,-1){1}}
	\put(18,23)          {\line(1,0){8}}
	\put(0,0)            {\line(1,1){2}}
	\put(0,25)           {\line(1,-1){2}}
	\put(28,0)           {\line(-1,1){2}}
	\put(28,25)          {\line(-1,-1){2}}
%
%   
% d'igitos
%
%   
       \put(2,20)  {\ABACOD{#1}}
       \put(2,15)  {\ABACOD{#2}}
       \put(2,10)  {\ABACOD{#3}}
       \put(2,5)   {\ABACOD{#4}}
%      
\end{picture}
    } % \ABACO
    
 
\renewcommand{\thetable}{\Roman{table}}
\newcommand{\x} {$\bullet$}


\begin{document}

% CAPA
\begin{titlepage}
  \thispagestyle{empty}
  \begin{center} {\large \textbf{UNIVERSIDADE~ESTADUAL~DE~CAMPINAS}} \end{center}
  \begin{center} {\large INSTITUTO~DE~COMPUTAÇÃO}                    \end{center}
  \vspace{0.1cm}
  \begin{center}
    \begin{minipage}[tl]{31mm}
      \ABACO{1}{9}{6}{9}{1}
    \end{minipage}
  \end{center}
  \vspace{0.3cm}
  \begin{center} 
    {\large \textsc{Servidor de Agenda baseado em socket UDP
      }} 
    \\\vspace{0.5cm}
    {\textsl{Relatório do segundo laboratório de MC823}}
    \\\vspace{1cm}
    \begin{tabular}{ll}
      \textbf{Aluno}:        Marcelo~Keith~Matsumoto   &  \textbf{RA}:       085937 \\
      \textbf{Aluno}:        Tiago~Chedraoui~Silva    &   \textbf{RA}:       082941 \\
      
    \end{tabular}
  \end{center}
  \vspace{0.5cm}

  \begin{abstract}
  \end{abstract}

\end{titlepage} 
  % Sumário
  \tableofcontents

\newpage


% -----------------------------------------------------------------------------%
\section{Objetivo}
% -----------------------------------------------------------------------------%
O objetivoterceiro projeto de laboratório de teleprocessamento e redes é
comparar duas implementações distintas do modelo cliente-servidor: Java
RMI (Remote Method Invocation) e socket TCP. 
É de suma importância que utilizando a tecnologia Java RMI, cria-se
uma agenda, para possibilitar uma comparação
com a mesma agenda em socket TCP, criada anteriormente no projeto 1.

\subsection{Teoria}
Java RMI é uma das abordagens da tecnologia Java, construída para prover as
funcionalidade de uma plataforma de objetos distribuídos e com sua API (Application
Programming Interface) especificada pelo pacote java.rmi e seus subpacotes. A
arquitetura RMI viabiliza a interação de um objeto ativo em uma máquina virtual Java com
objetos de outras máquinas virtuais Java.

Para o funcionamento de uma plataforma de objetos distribuídos, a mesma precisa
da disponibilidade das seguintes ações: localização de objetos remotos (realizado pelo
RMI registry), comunicação com objetos remotos (controlado pelo Java RMI) e
carregamento de definições de classes para objetos móveis.

Para o desenvolvimento de uma aplicação cliente-servidor em Java RMI, são
necessários dois programas separados, um para o cliente e um para o servidor, e a
execução do serviço de registro de RMI (RMI registry). Um servidor, em geral, instancia
objetos remotos, referencia estes objetos e liga-os em uma determinada porta através de
um bind, aguardando nesta porta os clientes invocarem os métodos destes objetos. Um
cliente, em geral, referência remotamente um ou mais objetos remotos de um servidor, e
invoca os métodos destes objetos. Os mecanismos pelos quais o cliente e o servidor se
comunicam e trocam dados são fornecidos pelo Java RMI. O serviço de registro de RMI é
uma implementação de um serviço de nomes para RMI, no qual cada serviço
disponibilizado na plataforma é registrado através de um nome de serviço, ou seja, uma
string única para cada objeto o qual implementa serviços em RMI.

% -----------------------------------------------------------------------------%
\section{Servidor de agenda}
% -----------------------------------------------------------------------------%
O sistema implementado, uma agenda distribuída, se baseia numa comunicação
cliente-servidor. Nele o servidor possui todas as informações da
agenda que estão armazenadas em um banco de dados,
assim como as opções de interações com os dados que são apresentadas
aos clientes em formas de um menu.
O cliente só escolhe alguma opção de interação com os dados de
acordo com menu.


\subsection{Menu inicial}
No menu inicial pode-se:

\begin{itemize}
\item Logar
\item Criar novo usuário
\item Sair
\end{itemize}

\subsubsection{Login}
O servidor pede ao usuário o nome de usuário, caso o nome estiver no
banco de dados ele pede uma senha que é comparada ao valor do banco de
dados, se o usuário não existir é avisado sobre a inexistência, se a
senha não conferir é avisado que a senha não confere, caso contrário o
usuário consegue logar no sistema, e o servidor recupera sua agenda (cada
usuário possui sua agenda).


\subsubsection{Novo usuário}
O servidor pede um nome de usuário, o servidor verifica se o nome já
não existe, se não existir pede a senha e armazena o usuário no
sistema, assim como cria uma agenda vazia para o mesmo.

\subsection{Menu usuário}
Dentre as possibilidades de interações para um usuário logado tem-se:

\begin{itemize}
\item Inserção de um compromisso que possui um nome, dia, hora, e minuto. 
\item Remoção de um compromisso através de seu nome
\item Pesquisa de compromisso por dia
\item Pesquisa de compromisso por dia e hora
\item Ver todos os compromisso de mês de abril
\end{itemize}

\subsubsection{Inserção de compromisso}
O usuário deve fornecer o nome do compromisso, o dia, a hora e o
minutos em que ele ocorrerá.
Caso o compromisso seja possível de ser alocado o servidor avisa com
um ``OK'', se não for possível também é avisado de tal impossibilidade.
Um compromisso é inserido ordenado na agenda se não existir um
compromisso com mesmo horário.

\subsubsection{Remoção de compromisso}
O usuário deve fornecer o nome do compromisso que deve ser removido.
Caso o compromisso seja encontrado ele é removido, caso contrário é
dito que tal compromisso não existe.
Se existirem dois compromissos de mesmo nome, o primeiro é removido.
Logo é esperado que compromissos possuam nomes diferentes.


\subsubsection{Pesquisas}
O servidor faz um requerimento interativo, ou seja, se for selecionado
a pesquisa por dia e hora, o servidor pergunta primeiramente o dia e
depois a hora. Logo, é uma pesquisa em etapas no qual o servidor
interage com nosso usuário.

% -----------------------------------------------------------------------------%
\section{Ambiente de implementação}
% -----------------------------------------------------------------------------%


  \section{Tempos de comunicação e total}


  \subsection{Comparação de tecnologias}

\section{Conclusão}

% ******************************************************
% REFERENCIAS BIBLIOGRÁFICAS
% ******************************************************
% \section{Referências}
\bibliographystyle{plain}
\begin{small}
  \bibliography{referencias}
\end{small}
\newpage
\section{Anexo}

\lstset{
  basicstyle=\tiny\ttfamily, % Standardschrift
  numbers=left,               % Ort der Zeilennummern
  numberstyle=\scriptsize,          % Stil der Zeilennummern
  numbersep=5pt,              % Abstand der Nummern zum Text
  tabsize=2,                  % Groesse von Tabs
  extendedchars=true,         %
  breaklines=true,            % Zeilen werden Umgebrochen
  classoffset=0,
  keywordstyle=\color{blue},
  classoffset=1,
  morekeywords={factor},keywordstyle=\color{red},
  classoffset=0,
  stringstyle=\color{green}\ttfamily, % Farbe der String
  showspaces=false,           % Leerzeichen anzeigen ?
  showtabs=false,             % Tabs anzeigen ?
  xleftmargin=17pt,
  framexleftmargin=17pt,
  framexrightmargin=5pt,
  framexbottommargin=4pt,
  showstringspaces=false      % Leerzeichen in Strings anzeigen ?        
}
\lstloadlanguages{Java}

\begin{multicols}{2}
%\lstinputlisting[language=java,caption={Agenda }]{code/rel/agenda.c}
\end{multicols}
\end{document}
