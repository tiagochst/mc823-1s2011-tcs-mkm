\documentclass[10pt,a4paper]{article}
\usepackage[T1]{fontenc}
\usepackage[brazil]{babel}
\usepackage[utf8]{inputenc}


\usepackage{ae,aecompl}
\usepackage{pslatex}
\usepackage{epsfig}
\usepackage{geometry}
\usepackage{url}
\usepackage{textcomp}
\usepackage{ae}
\usepackage{subfig}
\usepackage{indentfirst}
\usepackage{textcomp}
\usepackage{color}
\usepackage{setspace}
\usepackage{verbatim}

\usepackage{listings}
\lstset{language=C}
\usepackage{listings}
\lstset{
  basicstyle=\footnotesize\ttfamily, % Standardschrift
  numbers=left,               % Ort der Zeilennummern
  numberstyle=\tiny,          % Stil der Zeilennummern
  % stepnumber=2,               % Abstand zwischen den Zeilennummern
  numbersep=5pt,              % Abstand der Nummern zum Text
  tabsize=2,                  % Groesse von Tabs
  extendedchars=true,         %
  breaklines=true,            % Zeilen werden Umgebrochen
  classoffset=0,
  keywordstyle=\color{blue},
  classoffset=1,
  morekeywords={factor},keywordstyle=\color{red},
  classoffset=0,
  % frame=b,         
  % keywordstyle=[1]\textbf,    % Stil der Keywords
  % keywordstyle=[2]\textbf,    %
  % keywordstyle=[3]\textbf,    %
  % keywordstyle=[4]\textbf,   \sqrt{\sqrt{}} %
  stringstyle=\color{green}\ttfamily, % Farbe der String
  showspaces=false,           % Leerzeichen anzeigen ?
  showtabs=false,             % Tabs anzeigen ?
  xleftmargin=17pt,
  framexleftmargin=17pt,
  framexrightmargin=5pt,
  framexbottommargin=4pt,
  % backgroundcolor=\color{lightgray},
  showstringspaces=false      % Leerzeichen in Strings anzeigen ?        
}
\lstloadlanguages{C
}

% Definindo o espaçamento entre linhas para 1.5
% Relatório parcial deve ter espaçamento simples
\linespread{1.5}

\geometry{ 
  a4paper,	% Formato do papel
  tmargin=40mm,	% Margem superior
  bmargin=40mm,	% Margem inferior
  lmargin=20mm,	% Margem esquerda
  rmargin=20mm,	% Margem direita
  footskip=20mm	% Espaço entre o rodapé e o fim do texto
}
%  ABACO -- Conjunto de macros para desenhar o 'abaco

%  Desenho original de Hans Liesenberg

%  Macros de Tomasz Kowaltowski

%  DCC -- IMECC -- UNICAMP

%  Mar,co de 1988  --  Vers~ao 1.0

% Ajustado para LaTeX da SUN -- Mar,co de 1991

% ---------------------------------------------------------

%  Chamada:   \ABACO{d1}{d2}{d3}{d4}{esc}
%             com:  di's -- os quatro d'igitos;
%	           esc  -- fator de escala

% ---------------------------------------------------------

%  DEFINI,C~OES AUXILIARES

% ---------------------------------------------------------


%  Forma o d'igito pequeno (0 ou 1)

\newcommand{\ABACODP}[1]{%
%
\thicklines
%    
\begin{picture}(8,0)
    \ifcase#1{   %  caso 0
       \put(0,0)    {\line(1,0){4}}
       \multiput(5,0)(2,0){2}{\oval(2,4)}}
    \or{         %  caso 1
       \put(2,0)    {\line(1,0){4}}
       \multiput(1,0)(6,0){2}{\oval(2,4)}}
    \fi
\end{picture}
    } % \ABACODP

% Forma o d'igito grande (0 a 4)

\newcommand{\ABACODG}[1]{%
%
\thicklines
%    
\begin{picture}(14,0)
    \ifcase#1{   % caso 0
       \multiput(1,0)(2,0){5}{\oval(2,4)}}
       \put(10,0)   {\line(1,0){4}}
    \or{         % caso 1
       \multiput(1,0)(2,0){4}{\oval(2,4)}}
       \put(8,0)   {\line(1,0){4}}
       \put(13,0)   {\oval(2,4)}
    \or{         % caso 2
       \multiput(1,0)(2,0){3}{\oval(2,4)}
       \put(6,0)   {\line(1,0){4}}
       \multiput(11,0)(2,0){2}{\oval(2,4)}}
    \or{         % caso 3
       \multiput(1,0)(2,0){2}{\oval(2,4)}
       \put(4,0)   {\line(1,0){4}}
       \multiput(9,0)(2,0){3}{\oval(2,4)}}
    \or{         % caso 4
       \put(1,0)  {\oval(2,4)}}
       \put(2,0)   {\line(1,0){4}}
       \multiput(7,0)(2,0){4}{\oval(2,4)}
    \fi
\end{picture}
    } % \ABACODG
       
% Forma um d'igito (0 a 9)

\newcommand{\ABACOD}[1]{%
%
    \ifnum#1>9
       \errmessage{#1: Argumento invalido para ABACO}
    \fi
    \ifnum#1<0
       \errmessage{#1: Argumento invalido para ABACO}
    \fi
%
\begin{picture}(24,0)
%    
    \ifnum#1<5
       \put(16,0) {\ABACODP{0}}
    \else   
       \put(16,0) {\ABACODP{1}}
    \fi
%    
    \ifnum#1<5
       \put(0,0)  {\ABACODG{#1}}
    \else
       \ifcase#1\or \or \or \or
          \or  \put(0,0)  {\ABACODG{0}}
          \or  \put(0,0)  {\ABACODG{1}}
          \or  \put(0,0)  {\ABACODG{2}}
          \or  \put(0,0)  {\ABACODG{3}}
          \or  \put(0,0)  {\ABACODG{4}}
       \fi
    \fi   
\end{picture}
    } % \ABACOD
    
% -------------------------------------------------

%  DEFINI,C~AO PRINCIPAL
    
\newcommand{\ABACO}[5]{%
    \setlength{\unitlength}{#5mm}
%
    \thinlines
%   
\begin{picture}(28,25)
%   
% moldura
%
% externa
%
        \put(0,0)            {\line(0,1){25}}
        \put(0,0)            {\line(1,0){28}}
        \put(28,0)           {\line(0,1){25}}
        \put(0,25)           {\line(1,0){28}}
% interna
        \put(2,2)            {\line(0,1){21}}
	\put(26,2)           {\line(0,1){21}}
	\put(16,2)           {\line(0,1){21}}
	\put(18,2)           {\line(0,1){21}}
	\put(2,2)            {\line(1,0){14}}
	\put(16,2)           {\line(1,-1){1}}
	\put(17,1)           {\line(1,1){1}}
	\put(18,2)           {\line(1,0){8}}
	\put(2,23)           {\line(1,0){14}}
	\put(16,23)          {\line(1,1){1}}
	\put(17,24)          {\line(1,-1){1}}
	\put(18,23)          {\line(1,0){8}}
	\put(0,0)            {\line(1,1){2}}
	\put(0,25)           {\line(1,-1){2}}
	\put(28,0)           {\line(-1,1){2}}
	\put(28,25)          {\line(-1,-1){2}}
%
%   
% d'igitos
%
%   
       \put(2,20)  {\ABACOD{#1}}
       \put(2,15)  {\ABACOD{#2}}
       \put(2,10)  {\ABACOD{#3}}
       \put(2,5)   {\ABACOD{#4}}
%      
\end{picture}
    } % \ABACO
    
 
\renewcommand{\thetable}{\Roman{table}}
\newcommand{\x} {$\bullet$}


\begin{document}

% CAPA
\begin{titlepage}
  \thispagestyle{empty}
  \begin{center} {\large \textbf{UNIVERSIDADE~ESTADUAL~DE~CAMPINAS}} \end{center}
  \begin{center} {\large INSTITUTO~DE~COMPUTAÇÃO}                    \end{center}
  \vspace{0.1cm}
  \begin{center}
    \begin{minipage}[tl]{31mm}
      \ABACO{1}{9}{6}{9}{1}
    \end{minipage}
  \end{center}
  \vspace{0.3cm}
  \begin{center} 
    {\large \textsc{Servidor de Agenda baseado em socket TCP
      }} 
    \\\vspace{0.5cm}
    {\textsl{Relatório do primeiro laboratório de MC823}}
    \\\vspace{1cm}
    \begin{tabular}{ll}
      \textbf{Aluno}:        Marcelo~Keith~Matsumoto   &  \textbf{RA}:       085937 \\
      \textbf{Aluno}:        Tiago~Chedraoui~Silva    &   \textbf{RA}:       082941 \\
      
    \end{tabular}
  \end{center}
  \vspace{0.5cm}

  \begin{abstract}
O TCP (Transmission Control Protocol) é um protocolo do nível da
camada de transporte~\cite{Kurose05}. Algumas de suas características são 
transferência garantida, controle de congestão, ser orientado à
conexão e controle de fluxo.

Utilizando esse protocolo desenvolveu-se uma aplicação distribuída
que simulava uma agenda para o mês de
abril no qual avaliou-se a interação entre servidor e cliente através
de diversas funções (inserção de dados na agenda, recuperação de
dados, remoção de dados) em
termos de tempo de comunicação e tempo total.
  \end{abstract}

\end{titlepage} 
  % Sumário
  \tableofcontents
\newpage


% -----------------------------------------------------------------------------%
\section{Introdução}
% -----------------------------------------------------------------------------%
O TCP (Transmission Control Protocol) é um protocolo do nível da
camada de transporte~\cite{Kurose05}. Dentre usas principais características temos:
\begin{description}
\item[Orientado à conexão]  Cliente e o servidor trocam pacotes de
  controle entre si antes de enviarem os pacotes de dados. Isto é chamado de procedimento de
  estabelecimento de conexão (handshaking). 
\item[Transferência garantida] Dados trocados são livres de erro, o que é conseguido a partir de
  mensagens de reconhecimento e retransmissão de pacotes. 
\item[Controle de fluxo] Assegura que nenhum dos lados da comunicação envie pacotes rápido demais, pois uma aplicação em um
  lado pode não conseguir processar a informação na velocidade que está
  recebendo.
\item[controle de congestão] Ajuda a prevenir congestionamentos na
  rede.
\item[Fim-a-fim ] Conexão é sempre entre o host emissor e o host receptor.

\end{description}

Cada um dos segmentos da camada transporte tem em seu cabeçalho campos
denominado número de portas que indicam a qual processo o mesmo deve
ser entregue, existindo um número de porta do emissor e o número de porta do
receptor. 

Possuindo as duas portas, pode-se realizar uma conexão entre elas
conhecida por socket. Com o socket é possível diferenciar os canais
utilizados por cada processo de aplicação.

Os segmentos TCP, através do protocolo IP, são entregues a sistemas terminais, cada um
identificado por seu endereço IP. E o protocolo TCP cuida de repassar
os dados à cada processo de aplicação de acordo com porta especificada no
cabeçalho do segmento.

Devido a importância do protocolo, este laboratório tem o objetivo de
medir o tempo total e de comunicação de uma conexão TCP entre um
cliente e um servidor.

% -----------------------------------------------------------------------------%
\section{Servidor de agenda}
% -----------------------------------------------------------------------------%
O sistema implementado, uma agenda distribuída, se baseia numa comunicação
cliente-servidor. Nele o servidor possui todas as informações da
agenda que estão armazenadas em um banco de dados,
assim como as opções de interações com os dados que são apresentadas
aos clientes em formas de um menu.
O cliente só escolhe alguma opção de interação com os dados de
acordo com menu.


\subsection{Menu inicial}
No menu inicial pode-se:

\begin{itemize}
\item Logar
\item Criar novo usuário
\item Sair
\end{itemize}

\subsubsection{Login}
O servidor pede ao usuário o nome de usuário, caso o nome estiver no
banco de dados ele pede uma senha que é comparada ao valor do banco de
dados, se o usuário não existir é avisado sobre a inexistência, se a
senha não conferir é avisado que a senha não confere, caso contrário o
usuário consegue logar no sistema, e o servidor recupera sua agenda (cada
usuário possui sua agenda).


\subsubsection{Novo usuário}
O servidor pede um nome de usuário, o servidor verifica se o nome já
não existe, se não existir pede a senha e armazena o usuário no
sistema, assim como cria uma agenda vazia para o mesmo.

\subsection{Menu usuário}
Dentre as possibilidades de interações para um usuário logado tem-se:

\begin{itemize}
\item Inserção de um compromisso que possui um nome, dia, hora, e minuto. 
\item Remoção de um compromisso através de seu nome
\item Pesquisa de compromisso por dia
\item Pesquisa de compromisso por dia e hora
\item Ver todos os compromisso de mês de abril
\end{itemize}

\subsubsection{Inserção de compromisso}
O usuário deve fornecer o nome do compromisso, o dia, a hora e o
minutos em que ele ocorrerá.
Caso o compromisso seja possível de ser alocado o servidor avisa com
um ``OK'', se não for possível também é avisado de tal impossibilidade.
Um compromisso é inserido ordenado na agenda se não existir um
compromisso com mesmo horário.

\subsubsection{Remoção de compromisso}
O usuário deve fornecer o nome do compromisso que deve ser removido.
Caso o compromisso seja encontrado ele é removido, caso contrário é
dito que tal compromisso não existe.
Se existirem dois compromissos de mesmo nome, o primeiro é removido.
Logo é esperado que compromissos possuam nomes diferentes.


\subsubsection{Pesquisas}
O servidor faz um requerimento interativo, ou seja, se for selecionado
a pesquisa por dia e hora, o servidor pergunta primeiramente o dia e
depois a hora. Logo, é uma pesquisa em etapas no qual o servidor
interage com nosso usuário.

% -----------------------------------------------------------------------------%
\section{Ambiente de implementação}
% -----------------------------------------------------------------------------%
O sistema de agenda foi implementado e executado nos seguintes sistemas operacionais :
\begin{itemize}
\item FC14 - Fedora Laughlin Linux 2.6.35.11
\end{itemize}

O sistema de agenda foi implementado na linguagem C.
Para o armazenamento dos dados, utilizou-se arquivos. Cada usuário
possui um arquivo, a sua agenda, no qual armazena-se o nome do
compromisso, o dia, a hora e o minuto do mesmo.
O sistema lê esse arquivo quando o usuário loga e transfere-o à
memória principal, e a cada alteração na agenda o servidor atualiza as informações dos arquivos.

O servidor aceita diversas conexões de clientes, funcionando
perfeitamente para interações em diferentes agendas, pois cada
cliente possui além de um processo único, que foi criado em um fork,
possui um ponteiro para sua agenda. Assim, o servidor consegue
alterar todas as agendas independentemente.

 O nosso sistema, além disso,apresenta
transparência ao usuário. Os tipos de transparência a serem destacados
são:

\begin{description}
\item [Acesso:] Esconde as diferenças nas representações de dados e na invocação de funções ou métodos para facilitar a comunicação entre objetos da rede.
\item [Localização:] Esconde o local em que o objeto se encontra.
\item [Concorrência:] Esconde como as atividades são coordenadas entre os objetos para obter consistência em um nível mais alto.
\end{description}


Listaremos a seguir algumas funções implementadas de interação:
\begin{itemize}
\item  Funções de interação com o banco de dados são:
  \begin{lstlisting}
    /* Encontra usuario que ja esta cadastrado no servidor e verifica
    senha*/
    int findUser(char nome[], char pwd[]);
    
    /* Insere novo usuario (nome e senha) no banco de dados*/
    int newUser(char nome[], char senha[]);
    
    /*Carrega agenda de usuario*/
    int loadCal(User *user);
    
    /*Salva agenda*/
    int saveCal(User *user);
  \end{lstlisting}

\item Funções de interação com a agenda  são:

  \begin{lstlisting}
    /*Cria agenda para usuario*/
    User * agenda_init(char nome[]);

    /*Apaga agenda da memoria principal do servidor */
    void user_destroy(User *u);

    /*Insere compromisso na agenda*/
    int set_task(int dia,int hora, int min,char task[], User *u);

    /*Cria compromisso */
    Agenda * task_init(int dia,int hora, int min,char task[]);

    /*Retorna compromissos do mes de abril*/
    int verMes(int new_fd, User *u);

    /*Retorna compromissos do mes de abril em determinado dia*/
    int verDia(int new_fd, User *u, int dia);
    
    /*Retorna compromissos do mes de abril em determinado dia e
    determinada hora*/
    int verHora(int new_fd, User *u, int dia, int hora);

    /*Remove compromisso da agenda pelo nome*/
    int delTask( User *u, char nome[]);

    /*Comapara data de compromissos*/
    int compData(Agenda *newTasks,Agenda *tasks);

  \end{lstlisting}

\item Funções de interação servidor-cliente criadas foram:
  \begin{lstlisting}

    /*Envia mensagem ao cliente*/
    void sendStr(int sockfd, char str[]);

    /*Le mensagem do cliente*/
    int leOpcao(struct sockaddr_storage their_addr, int sockfd);
    
    /*Apresenta opcoes de login ou criacao novo usuario*/
    void menu(int new_fd, struct sockaddr_storage their_addr);
    
    /*Apresenta opcoes de interacao com agenda*/
    void menu2(int new_fd, struct sockaddr_storage their_addr, User *user);

    /*Envia mensagem para servidor*/
    void envia_pct( int sockfd, char s[], int size){

    \end{lstlisting}


  \end{itemize}




  \section{Tempos de comunicação e total}
  O round-trip time (RTT) é o tempo que leva-se para um sinal ser
  enviado mais o tempo que se leva para receber um acknowledgment que o
  sinal foi recebido. A ferramenta administrativa para as redes de
  computadores denominada ``Ping''~\cite{Ping} é usada para testar se um host é alcançável e para
  medir o RTT para mensagens enviadas do host remetente para o
  destinatário.
  Utilizado o ping nas máquinas nos quais servidor e cliente foram
  usadas obtivemos os valores apresentados na tabela \ref{pingtime}.

  \begin{table}[h!]
    \begin{center}
        \begin{tabular}{lr}
          \multicolumn{1}{c}{Valor} & \multicolumn{1}{c}{Tempo}\\
          \hline
          Max & 0.688 ms\\
          Min & 0.309 ms\\
          Média & 0.513 ms \\
          Desvio & 0.086 ms
        \end{tabular}
    \end{center}
    \vspace{-5mm}
    \caption{Tempo do ping}        \label{pingtime}
  \end{table}

  Posteriormente,  foi calculado o tempo que o servidor leva para se
  conectar com o servidor, como é mostrado na tabela \ref{conntime}. Os valores foram obtidos de um
  conjunto de 100 medições.

  \begin{table}[h!]
    \begin{center}
        \begin{tabular}{lr}
          \multicolumn{1}{c}{Valor} & \multicolumn{1}{c}{Tempo}\\
          \hline
          Max & 0.634 ms\\
          Min & 0.192 ms\\
          Média & 0.517 ms \\
          Desvio & 0.081 ms
        \end{tabular}
    \end{center}
    \vspace{-5mm}
    \caption{Tempo de conexão com o servidor}        \label{conntime}
  \end{table}

Aplicamos o cálculo de tempo ao programa principal de
  forma a obtermos o tempo total,  tempo de comunicação e os tempos da
  execução de cada função.
  Para o tempo total, no cliente pega-se o tempo antes do primeiro send e após o último recv.
  Para o tempo de comunicação, pega-se o tempo total e subtrai-se o tempo de processamento do servidor,
  que é depois do primeiro recv e antes do último send.

  Para o tempo total das funções obteu-se o tempo de inserir um
  compromisso, remover o compromisso, ver a agenda do mês, ver a agenda
  de um dia, ver a agenda de uma hora. Os dados e os testes estão
  exemplificados nas tabelas \ref{op1}, \ref{op2}, \ref{op3},
  \ref{op4} e \ref{op5}.

  O resultado obtido para 100 valores foi:

  \begin{table}[h!]
    \begin{center}

      \subfloat[Tempo total]{
        \begin{tabular}{lr}
          \multicolumn{1}{c}{Valor} & \multicolumn{1}{c}{Tempo}\\
          \hline
          Max & 185.351 ms\\
          Min & 45.577 ms\\
          Média & 55.090 ms \\
          Desvio & 0.743 ms
        \end{tabular}
        \label{total1}
      }
      \hspace{30mm}
      \subfloat[Tempo de comunicação]{
        \begin{tabular}{lr}
          \multicolumn{1}{c}{Valor} & \multicolumn{1}{c}{Tempo}\\
          \hline
          Max & 1.483 ms\\
          Min & 0.500 ms\\
          Média & 0.725 ms \\
          Desvio & 0.076 ms
        \end{tabular}
        \label{comu1}
      }

    \end{center}
    \vspace{-5mm}
    \caption{Conexão e fechamento de conexão com servidor} \label{op1}
  \end{table}


  \begin{table}[h!]

    \begin{center}

      \subfloat[Tempo total]{
        \begin{tabular}{lr}
          \multicolumn{1}{c}{Valor} & \multicolumn{1}{c}{Tempo}\\
          \hline
          Max & 114.572 ms\\
          Min & 32.248 ms\\
          Média & 36.782 ms \\
          Desvio & 0.069 ms
        \end{tabular}
        \label{total2}
      }
      \hspace{30mm}      
     \subfloat[Tempo de comunicação]{
        \begin{tabular}{lr}
          \multicolumn{1}{c}{Valor} & \multicolumn{1}{c}{Tempo}\\
          \hline
          Max & 0.837 ms\\
          Min & 0.386 ms\\
          Média & 0.705 ms \\
          Desvio & 0.001 ms
        \end{tabular}
        \label{comu2}
      }

    \end{center}
    \vspace{-5mm}
    \caption{Conexão, login na conta, inserção de compromisso e
      fechamento de conexão com servidor} \label{op2}
  \end{table}


  \begin{table}[h!]
    \begin{center}
      \subfloat[Tempo total]{
        \begin{tabular}{lr}
          \multicolumn{1}{c}{Valor} & \multicolumn{1}{c}{Tempo}\\
          \hline
          Max & 214.583 ms\\
          Min & 37.724 ms\\
          Média & 47.599 ms \\
          Desvio & 0.468 ms
        \end{tabular}
        \label{total3}
      }
      \hspace{30mm}
      \subfloat[Tempo de comunicação]{
        \begin{tabular}{lr}
          \multicolumn{1}{c}{Valor} & \multicolumn{1}{c}{Tempo}\\
          \hline
          Max & 0.851 ms\\
          Min & 0.515 ms\\
          Média & 0.715 ms \\
          Desvio & 0.004 ms
        \end{tabular}
        \label{comu3}
      }
    \end{center}
    \vspace{-5mm}
    \caption{Conexão, login na conta, ver compromissos de determinado
      dia, hora e minuto e fechamento de conexão com servidor} \label{op3}
 \end{table}

  \begin{table}[h!]
    \begin{center}

      \subfloat[Tempo total]{
        \begin{tabular}{lr}
          \multicolumn{1}{c}{Valor} & \multicolumn{1}{c}{Tempo}\\
          \hline
          Max & 168.404 ms\\
          Min & 32.763 ms\\
          Média & 38.612 ms \\
          Desvio & 0.412 ms
        \end{tabular}
        \label{total4}
      }
      \hspace{30mm}
      \subfloat[Tempo de comunicação]{
        \begin{tabular}{lr}
          \multicolumn{1}{c}{Valor} & \multicolumn{1}{c}{Tempo}\\
          \hline
          Max & 1.472 ms\\
          Min & 0.523 ms\\
          Média & 0.727 ms \\
          Desvio & 0.002 ms
        \end{tabular}
        \label{comu4}
      }

    \end{center}
    \vspace{-5mm}
    \caption{Conexão, login na conta, ver compromissos de determinado
      dia e hora e fechamento de conexão com servidor} \label{op4}
  \end{table}

  \begin{table}[h!]
    \begin{center}

      \subfloat[Tempo de total]{
        \begin{tabular}{lr}
          \multicolumn{1}{c}{Valor} & \multicolumn{1}{c}{Tempo}\\
          \hline
          Max & 76.481 ms\\
          Min & 28.966 ms\\
          Média & 32.334 ms \\
          Desvio & 0.215 ms
        \end{tabular}
        \label{total5}
      }
      \hspace{30mm}
      \subfloat[Tempo de comunicação]{
        \begin{tabular}{lr}
          \multicolumn{1}{c}{Valor} & \multicolumn{1}{c}{Tempo}\\
          \hline
          Max & 0.905 ms\\
          Min & 0.504 ms\\
          Média & 0.716 ms \\
          Desvio & 0.000 ms
        \end{tabular}
        \label{comu5}
      }

    \end{center}
    \vspace{-5mm}
    \caption{Conexão, login na conta, ver todos os compromissos do mês
      e fechamento de conexão com servidor} \label{op5}
  \end{table}

  
  Note que os tempos das tabelas \ref{total3}, \ref{total4} e
  \ref{total5} estão, respectivamente, em ordem decrescente de
  tamanho. Isso se deve ao fato de que a operação 3 (tabela
  \ref{total3}) o programa busca pelo dia, hora e minuto do
  compromisso, fazendo com que o número de comparações feitas seja
  maior que as operações \ref{total4} e \ref{total5}, tornando aquela
  operação mais lenta. O mesmo pode afirmar entre as operações 4 e 5,
  correspondentes às tabelas \ref{total4} e \ref{total5}, respectivamente.

\subsection{Comparação de tecnologias}
Como uma das características do TCP é a transferência garantida não foi
necessário uma análise de erros na entrega dos pacotes. No entanto, essa análise
deve ser feita se utilizássemos o protocolo UDP, através de um código
de verificação. Portanto o protocolo TCP possibilita uma diminuição do código.

Para que o nosso sistema distribuído fosse transparente ao usuário
diversas manipulações de dados foram realizadas no servidor, assim
algumas transparências, como localidade, acesso e concorrência foram
possíveis.

%Há também a necessidade de se verificar o pacote recebido veio do endereço correto, dado que
%na conexão UDP os dados são simplesmente enviados para o outro host.

Como o TCP é orientado à conexão, é criado um socket para cada cliente
se comunicar com o servidor. Utilizando a função fork temos um
processo que fica à escuta de novos pedidos de conexão e outro
processo trata da conexão com o cliente. 
Já o UDP não é orientado a
conexão, portando não há necessidade de se criar um socket para o
cliente, deve tratar-se o cliente de modo interativo.
Para o UDP é necessário verificar qual a fonte do pacote, enquanto o
TCP já cuida disso.

% não tenho certeza disso.
% Pelo fato de o TCP criar diversos processos,
%cada conexão faz uso do processamento do servidor de modo
%concorrente. Já o UDP como só trata um cliente por instante, ou seja,
%ele avalia o pacote de um cliente e trabalha nele enquanto lê o
%próximo segmento que chegará. 

Devido as diferentes características do UDP como, por exemplo, o não há tratamento de
erros e  não ser orientado a conexão, é esperado que ele tenha
um tempo inferior ao TCP. Mas, deve-se lembrar que os erros deverão
ser verificados pelo programa principal, assim como o ordenamento das mensagens.

O tempo de processamento (manipulação da agenda) teoricamente não deve
alterar pela utilização dos diferentes protocolos.

Portanto, é esperado o tempo total será alterado devido ao tempo de comunicação.

%Em termos de tempo total, temos que a espectativa de mudança será responsável pela
%parcela modificada resultante da comunicação, dado que o tempo total é resultante da soma do
%tempo de comunicação e do tempo tempo de processamento.

\section{Conclusão}

% ******************************************************
% REFERENCIAS BIBLIOGRÁFICAS
% ******************************************************
% \section{Referências}
\bibliographystyle{plain}
\begin{small}
  \bibliography{referencias}
\end{small}

\end{document}
